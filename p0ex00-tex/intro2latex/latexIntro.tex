\documentclass[a4paper, 18pt]{article}

\usepackage[utf8]{inputenc}
\usepackage[brazil]{babel}
\usepackage[lmargin=1cm, tmargin=1cm, rmargin=1cm, bmargin=1cm]{geometry}
\usepackage[]{amsmath, amsthm, amsfonts, amssymb, dsfont, mathtools}
\usepackage[]{graphicx}

\title{Introdução ao \LaTeX}
\author{Traian Matisi}
%\date{13 mar 2023}

\begin{document}

    \maketitle

    \section{Introdução}

    Comandos são escritos com contra-barra no início\\

\begin{itemize}

    \item [] documentclass\{\}
    \item [] usepackage\{\}
    \item [] usepackage\{\}
    \item [] ...
    \item [] usepackage\{\}

    title\{\}\\
    author\{\}\\
    date\{\}\\

    begin{}
    end{}


\end{itemize}
    

    Símbolos de matemática

\begin{itemize} % pode usar begin enumerate

    \item [i] Soma: $x+y$
    \item [ii] Diferença: $x-y$
    \item [iii] Produto: $x\cdot y$
    \item [iii] Produto: $x\times y$
    \item [iv] Divisão: $\frac{x}{y}$
    \item [iv] Divisão: $\dfrac{x}{y}$
    \item [iv] Divisão: $x\div y$
    \item [iv] Divisão: $x / y$
    \item [iv] Divisão: $x : y$

\end{itemize}

\end{document}